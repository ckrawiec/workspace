\documentclass{article}
\usepackage{graphicx}
\usepackage{float}

\begin{document}

\centerline{\large Progress on Examination of Anemometer and Accelerometer Data}
\centerline{May 17, 2014 - }


\section{Introduction}
State of the art in weak lensing?
	The light from distant galaxies is bent on its way to observers on earth by intervening mass. This bending gives us information about the composition and evolution of the universe. This effect is very small and thus its measurement requires good accuracy and precision. Unfortunately, telescopes on Earth deal with a handicap caused by the atmosphere that distant light rays encounter just before being received. The density and movement of these atmospheric layers act to blur the recorded images. This blurring can change the appearance of galaxy shapes. This is clearly problematic for measuring distortion caused by gravitational lensing. To try and correct for the blurring caused by this effect and other imperfections of the telescope structure, one can look at the shapes of stars. Since stars are effectively point sources, any distortion therin can be attributed to this blurring. The size measured from stars on an images is called the point spread function (PSF). One can deconvolve this PSF from the image, but its information can still leak into weak lensing measurements. 
	Any ellipticity contribution from the PSF can be much larger than the ellipticity induced by gravitational lensing. 
	

Dark Energy Survey goals - psf size.
	Dark energy is characterized by the parameter w. To measure w to an accuracy of 5*% using weak lensing, the measured shear must be accurate to ***. 
	The Dark Energy Survey is a five-year ** square degree survey which uses a 500 Megapixel camera located on the Blanco 1.4 meter telescope on Cerro Tololo in Chile. 
	
What kind of ellipticity is acceptable.
Current/at the time wind screen suggestions.
The Dark Energy Survey aims to .... sensitivities.
What frequencies are most important?
Getting rid of noise
Accelerometer sensitivity vs. Guider sensitivity - which one to look at?
How bad is too bad?
Speed recommendations - do they differ with official guidelines

--summary of work presented at Brighton meeting 2014--
I looked at all of the (working) anemometer data that was available. This consisted of time periods between 4/21/14 to 5/15/14 and 9/17/14 to time of meeting. 
I used two values of interest from the device: w_speed (one direction) and magnitude (2 combined directions). I found the typical rms values to be between 0 and 2 m/s, with a handful of nights > 3m/s. 
There is an accelerometer fitted on the telescope (where?***) that records data at 500 Hz. I averaged these data down to 50 Hz in each of the 3 directions. This was then divided into chunks by exposure. To accound for the overall drift and gravity, I subtract a linear fit from the exposure chunk. I applied a butterworth highpass filter at 1 Hz in the hopes that the guider will account for motion below this frequency. This signal is then integrated twice to get the displacement incurred on the telescope. This movement was then compared to the anemometer data by matching up the exposure times. 	
	An example windy night (May 9) shows the upward trend of acceleration and displacement with wind speed starting fairly sharply around 3 m/s. 
	Acceleration data is shown for 3 separate weeks (one in May, September, and October), keeping in mind that it was very time consuming gathering and reducing the data in Chile remotely. The typical displacements for these time periods rarely exceed 1-1.2 microns, with a slight upward trend at wind speeds above 3-4 m/s, though there were not very many data points in these region. It is shown that the motion seen is generally below the ~0.25% ellipticity level (find/redo this math). 
	A selection of high motion exposures are shown, with one low wind speed and three that are 2-5 m/s rms wind speeds. 
	The telescope control system (tcs) and guider can give insight into motion below 1 Hz. Data are shown from two of the same weeks as before from Y2. Definite positive trends can be seen in the guider and tcs errors with high wind speed, specifically above 4 m/s. 
	In conclusion, we see motion, but the highest rms wind speeds produce <1% ellipticity at frequencies greater than 1 Hz. Thus, it would be rare for the wind to have a significant impact on image quality. The guider, however, shows potentially significant impact at frequencies less than or equal to 1 Hz. The wind also seems to be causing this motion almost entirely in right ascension. These trends can be further
 monitored using the telemetry database. 
---------

\section{Anemometer}
The anemometer is located on the top ring of the Blanco Telescope. It is a WindMaster that measures the speed and direction of the wind near the telescope.
Wind speed units are selected by the user but can be given in mph, KPH, ft/min, m/s, or knots. Klaus confirms they are in m/s.

%%%
\section{Motivation}
At the May 2014 Dark Energy Survey collaboration meeting, Kevin Reil from SLAC gave a presentation on the data available from the accelerometer mounted on the Blanco telescope. He noted that there were some interesting features such as large spikes and periodic motion. An anemometer had just been installed on the telescope in April and was functioning properly. After the meeting I made a plan to compare the wind speed from the anemometer to the shake from the accelerometer. 
	The accelerometer is run through LabView and thus it's output files need to be converted to a more pliable format. I did my analysis in Python. I wrote a script to read these LabView files, convert them to binary FITS tables, and put the voltage in the correct acceleration units. I wrote a separate script that splits the raw output by exposure. The time stamp from the accelerometer and the actual start time of each exposure may be off by a few seconds. This should not have a significant impact on the data. 
	
\section{Filtering}
	It is necessary to run a lowpass filter over the raw data to remove the drift and changes in gravity due to the pointing of the telescope. A highpass filter is also applied to cut down on noise. 
%%%

\section{Telemetry Database}
Anemometer data appears to exist from 2014-01-31 16:39:15.137823+00:00 to 2014-05-15 14:41:21.538939+00:00 in the telemetry database. 
The table is missing values from 2014-03-12 14:14:22.614837+00:00 to 2014-04-21 14:45:35.676678+00:00.
The columns of interest are magnitude and w\_speed. We believe magnitude to be a combination of two directions (U and V) where as w\_speed is in the 'W' direction. 
The environmental\_data table has dome\_wspeed values starting from 2013-05-17 20:16:45.127278+00:00 and ending on 2013-05-30 13:44:45.315716+00:00. I do not think dome\_wspeed is useful for this analysis anyway. 
The environmental\_data table has wind\_speed data from 2011-11-16 12:57:38.292394+00:00 to the present. This data is from an instrument outside the dome.
I will use the anemometer table for plotting wind speed data for the available nights. This data is accessible to me only through the telemetry viewer online (http://decam03.fnal.gov:8080/TVCTIO/app/Q/index). Exposure information for matching up data can be found through the DESDM database or the Exposure Browser (http://decam03.fnal.gov:8080/EXPO/Expo.html), which can also be used for sample SQL queries. 

\section{Plan}
My plan is to create a table including all of the anemometer data available split up into days/nights, and then by exposure. Some possible statistics for each block of data incclude: mean, maximum, minimum, rms. 
I will then select a set of nights that samples uncharacteristically high winds and also average night conditions. 
I will then request accelerometer data for these nights. 
I can compare/look for correlations between the two sets of data to see if any anomalies in the movement of the telescope could be caused by high winds. 

The next step would be to see if the spikes in accelerometer data are having any ill effects on the quality of the images produced. 

\section{Data}

Kevin Reil's talk at the DES meeting at NCSA showed one night's worth of accelerometer data. 
He included exposures 275242 (0.1 G spike), 275247 (periodic spikes), 275249, 275251, 275266, and 275416. 
The Exposure Browser lists these exposures on the night of 01-18-2014. 
Kevin made the data for this night publicly available, so I used the csv file (accel.csv) to make a python script that will plot any of the values in the file (simpleplot.py).
I created plots for each night of anemometer data to visually check for interesting nights. 
Some months seem to go on for stretches with values sticking around 0.01-0.02. There is usually a rise and fall in the first half of the day.
In May, however, the values go up to order 1, reaching up to 3 in some cases.

I found out from Tom Diehl that the anemometer was inside at Las Tacas before April, so I will only look at data from 4/21 to 5/15. 
I created a table (anemom042114\_051514.csv) that includes the time recorded, wind speed, direction, magnitude (do not know what this means yet), utc time, exposure id, exposure time, and the absolute median, absolute max, and rms for each day as well as each exposure.
 It also includes the beginning and ending time of the exposure in utc time for the purposes of matching data points.

Mike Warner has said he can provide accelerometer info for one night at a time, so I told him the nights of 5/8 and 5/9 seemed to be the windiest. 
Mike gave me a table of accelerometer rms data for ten hours on 5/9. 
I used this to compare the difference of the acceleration variances (in X,Y,and Z and for the total) with the rms magnitude per exposure. 
This is shown in Figure (?). 
There is a definite trend above \~2 m/s, mostly in the X-direction. 
I plotted these using rmsplots.py. 
We prescribed a cutoff ($1e-8 G^{2}$)for which values of the excess variance could potentially produce a 3\% effect on the image ellipticity. 
This was determined using a frequency of 2.5Hz based off of the graph of the telescope's modes provided by Mike Warner.


I have gained access to the Windows computer that logs the accelerometer data for chunks of time a couple of days or a week in length. 
I can transfer the compressed files to observer computer at DECamObserver@observer2.ctio.noao.edu. 
There is a directory there called /data_local/accel where I can keep scripts and .tdms files. 
There is a module called pyTDMS that will read in the .tdms files and dump them in a tuple containing the metadata and raw data. 
The 'All Data' lists exist for all three axes (labeled 0, 1, and 2: still have to find out if these correspond to x, y, and z in that order).
'All Data' lists contain raw data sampled at 500 Hz. 
I took this data and averaged every 10 samples to give 50 points per second.
The script tdms_to_fits.py will do this job and write each axis's data to a separate binary fits table. 
To separate the data by exposure, I created plot_voltage.py to match the dates by utc time in seconds. 
It's still unclear whether the initial time listed in the file is correct.
When converted to time stamp format, it the month, day, and time match the file name but the year is listed as 2080. 
Kevin Reil provided a python script (accel.py) that will take an ascii file provided by Mike and filter the data into rms values per 5 seconds of 500 Hz data.
I think I can take the high/low pass filter functions and the integration done in the analysis function (to get distances) and apply it to my data. 

\begin{figure}[h!]
  \centering
  \includegraphics[width=4in]{anemom_plots/windspeed2102014.png}
  \caption{Wind speed for 2014-02-10. This is the typical shape and values for Jan. 31st through March.}
\end{figure}

\begin{figure}[h!]
  \centering
  \includegraphics[width=4in]{anemom_plots/windspeed2262014.png}
  \caption{Wind speed for 2014-02-26. Sometimes there are weird periodic bursts like in this plot, but this is not the norm.}
\end{figure}

\begin{figure}[h!]
  \centering
  \includegraphics[width=4in]{anemom_plots/windspeed552014.png}
  \caption{Wind speeds for 2014-05-05. This looks like a typical night for the April/May days. There are usually one or two spikes like the one here.}
\end{figure}

\begin{figure}[h!]
  \centering
  \includegraphics[width=4in]{anemom_plots/windspeed592014.png}
  \caption{Wind speeds for 2014-5-9. An interesting shape in the wind speed. I guess it is just getting progressively windier? It looks like the dome is closed then at the place \
it drops back to ~zero. Looking at this plot I would guess the exposures start towards the end of the day and end where that drop is.}
\end{figure}

\begin{figure}[h!]
  \centering
  \includegraphics[width=4in]{anemom_plots/windspeed532014.png}
  \caption{Wind speeds for 2014-05-03. Looks like a fairly calm night compared to the others. It's easier to see here that even the down time speeds are higher than the ones in F\
eb/March.}
\end{figure}

\begin{figure}[h!]
  \centering
  \includegraphics[width=4in]{rms_vs_exp042114_051514.png}
  \caption{Rms wind speeds vs. exposure id. The green points are exposures >= 40s. The yellow are between 15 and 40s. The red are <15s. }
\end{figure}

\begin{figure}[h!]
  \centering
  \includegraphics[width=4in]{magrms_vs_speedrms.png}
  \caption{Comparison between the 'magnitude' and 'w_speed' columns for 5/9.}
\end{figure}

\section{Comparisons}

\begin{figure}[h!]
  \centering
  \includegraphics[width = 5in]{wind_speed_011814.png}
  \caption{External wind speeds for 2014-01-18.}
\end{figure}

\begin{figure}[h!]
  \centering
  \includegraphics[width=4in]{excessaccel_vs_magrms.png}
  \caption{Excess acceleration variance as a function of rms magnitude in all three spatial directions.}
\end{figure}

I then took those exposures which had variances above $1e-8 G^{2}$ and rms magnitudes below 2 m/s and highlighted them in some plots of telescope ellipticity.
In the telemetry database, I looked at the meanx2-meany2 of the telescope guider, the covxx-covyy from the image health focal plane table, and the gxx-gyy from tcs\_telemetry.
None of these plots show the outliers to be higher than usual. 
I also took a look to see if the low wind speed/excess acceleration points were made up of short exposures. 
All of the exposures < 15 s were part of that blob. 

\begin{figure}[h!]
  \centering
  \includegraphics[width=3in]{guider_ellips.png}
  \includegraphics[width=3in]{imagehealthfp_ellips.png}
  \includegraphics[width=3in]{tcs_telemetry_ellips.png}
  \caption{Various telescope ellipticities with accelerometer outliers in red.}
\end{figure}

\begin{figure}[h!]
  \centering
  \includegraphics[width=5in]{excessaccel_vs_magrms_byexp.png}
  \caption{Excess acceleration variance with exposures < 15 s highlighted in yellow.}
\end{figure}

\end{document}



