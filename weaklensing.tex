



Section 1
Basics of gravitational lensing

An object on the sky is gravitationally lensed when it passes behind another object.
A light ray from the background object follows an altered path due to the gravitational potential of the foreground one.
From an observer's perspective, the background object does not appear as it would unlensed - its observed position, size, and shape are altered.
The way these are altered depends on the nature of the foreground object's gravitational potential, which in turn depends on its mass distribution.
Thus, the mass distributions of bodies in the universe can be deduced using gravitational lensing. #- using images of distant galaxies?
#change object to body?

The formulas used to describe the bending of light due to a massive body are similar to those used in regular optics.
The background object is be referred to as the ``source'', and the foreground object the ``lens''.

Because the distances between the bodies involved in lensing are much larger than the distance along which lensing occurs, the three-dimensional mass distributions can be approximated as two-dimensional planes, the source plane and the lens plane. (meneghetti pg 17)

The mass distribution responsible for lensing is expressed as the surface mass density - the integral of the three-dimensional density along the line of sight. #(los)

The angle by which a ray of light is deflected from the unlensed path
depends on the surface mass density profile as well as the distances
between the source, lens, and observer. 

The mapping between the source plane and image plane is described by a
jacobian matrix. This matrix can be split up into its istropic and
anisotropic components. The isotropic transformation depends only on
the 'convergence' - equal to the surface mass density scaled by the
critical density. The critical density is a function of the distances
mentioned above. These distances are 'angular diameter' distances and
thus depend on cosmology. The convergence describes the change in size
induced by lensing. The anisotropic 'shear matrix' describes the
change in shape along two different axes in ellipticity space (???)

#lens diagram
#lens equation
#sigma equation
#deflection angle



#NFW profile???
#Cosmology - sigma crit (Dang), correlation functions that depend on matter power spectrum


Section 2
Weak lensing approximations


Section 3
Measuring Shear

#ellipticity
#tangential shear
#mass sheet degeneracy
#correlation fns

#different techniques? 

Section 4
Measuring Magnification

Surface brightness is conserved in lensing. Thus when a source's size is altered, its flux is magnified.

#breaking mass sheet degeneracy
#different techniques - number density, source size, correlation fns

Section 5
Systematic vs. Statistical errors

The key problem here is that lensing is not the only process by which galaxy properties become correlated. Galaxies at the same redshift can be aligned because of physical entanglements. Their shape and size can also be affected at the time of observation as well. Telescopic and atmospheric effects can produce spurious ellipticities that are correlated within the image, but are not truly correlated on the sky.

In the case of magnification, changes in number density could be caused instead by non-uniform photometry causing varying depth across the field. 

#PSF
#telescope infrastructure - leads to wind discussion


Section 6
State of the art and DES

#Goals
#DES meeting/trying to meet these goals
#Blanco/DECam specifics
