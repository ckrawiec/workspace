\title{Magnification and the Dark Energy Survey}
\author{
        Christina Krawiec
}
\date{\today}

\documentclass[12pt]{article}

\begin{document}
\maketitle

\begin{abstract}
This document attempts to outline the state of the art in lensing magnification as it pertains to ongoing efforts in the Dark Energy Survey.
\end{abstract}

\section{Introduction}



\section{Lyman Break Galaxies}
A Lyman break galaxy (LBG) is a star-forming galaxy located at high redshift that is identified by a characteristic drop in its UV rest-frame continuum due to attenuation by neutral Hydrogen surrounding it. 
These galaxies are classified by the band in which this ``dropout'' occurs, which varies due to redshift.
LBGs are good candidates for measuring magnification because of their large distances and steep number densities.
LBGs are about an order of magnitude more numerous than QSOs at the same flux level (?) and so provide a bigger sample of lensing sources.
Because of their steep relation between number density and magnitude, it is necessary to have deep enough observations in the right wavelength bands to gather a large sample. 
Dropouts in the u-band are typically the easiest to detect because they are located at a lower redshift (z~3).
Though magnification analyses have been done using g-dropouts ($z>4$) and farther using deeper exposures. 


\subsection{The Deep Lens Survey}
The Deep Lens Survey (DLS) is an ultra-deep multi-band sky survey with 5 fields in the Northern (2) and Southern (3) skies. 
Each field is $4 deg^{2}$ for a total area of $20 deg^{2}$.
DLS took images at Kitt Peak and CTIO (MOSAIC imager) in the BVRz filters.
The total exposure time is 12,000 s in BVz and 18,000 s in R. 
They look for B band dropouts at z = 4 and use ~12,000 LBGs after cuts.
? et al. make measurements of magnification in 7 redshift bins and find a signal at S/N > 20.


 - How do they make their LBG selections? Simulations? Just blind color cuts?

\subsection{CARS}

- Selection using HyperZ simulations
- Number counts as a function of magnitude/extrapolation to DES


\subsection{Dark Energy Survey}
As of August 2015, the Dark Energy Survey (DES) has completed two years of observations in addition to a science verification (SV) run.
This SV data covers 2-10 tilings over an area of $~300 deg^{2}$. 
The largest contiguous area of this set ($~150 deg^{2}$) overlaps with that of the South Pole Telescope and is labeled SPT-East (SPT-E).
According to *RedMagic paper*, the median depth (10$\sigma$ detection level(?)) of SV is 24 in g, 23.9 in r, 23 in i, 22.3 in z, and 20.8 in Y.
Compared to DLS and CARS, DES wins significantly in terms of covered area in the SV data alone. 
DES will cover $~2000 deg^{2}$(???) in varying levels of depth with Y1 data.
However, with its lower depth and lack of u-filter, DES may only be able to select ~1000 LBGs in SV. 

-Balrog, completeness?

\bibliographystyle{abbrv}
\bibliography{}

\end{document}